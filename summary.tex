\centerline{\bf This project summary will go in NSPIRES.  Put here for proofing} \medskip

\centerline{\bf Exploring the Critical Radius Between mini-Neptunes and super-Earths using Kepler} \medskip

We propose to combine a Bayesian reanalysis of short-period Kepler exoplanet transits with standard models of tidal theory in order to identify the planetary radius that separates rocky and gaseous exoplanets. We exploit the conventional assumption that gaseous planets dissipate orders of magnitude less tidal energy than rocky planets, leading to the expectation that the latter will be on circular orbits out to larger orbital periods. Preliminary dynamical simulations show that short period (2-10 days) gaseous bodies should be found with eccentricities near their primordial value, but rocky bodies are preferentially found at low eccentricity due to tidal circularization.  Thus, a study of the eccentricities of short-period planets can constrain the planetary radius of this transition.  The identification of the boundary between rocky and gaseous bodies, independently from mass measurements, is vital for understanding the planetary conditions needed to support life.

A lower limit to the orbital eccentricity can be calculated by comparing the difference between the modeled transit duration and the transit duration that would be seen if the orbit were circular.  To assess this difference, we analyze Kepler lightcurves using a purely geometric model that includes no assumptions about the orbital dynamics.  We cast our measurement of minimum eccentricity in terms of two model parameters, whose posterior distributions we explore using Markov Chain Monte Carlo methods, and two physical parameters (e.g. stellar mass and radius) that must be estimated from other means.  We have run a suite of simulations using a grid in transit depth, stellar brightness (lightcurve signal-to-noise), and the number of transits included in the model to gauge our sensitivity to these model parameters.  We validated this method on the confirmed exoplanet system Kepler 62-b, and successfully recovered the published results and expected parameter uncertainties. We propose here to extend this analysis to an ensemble of 1100+ KOIs that have been selected based upon their Kepler-reported periods and planetary radius.  This reanalysis will enable the first measurements of the boundary between gaseous and rocky exoplanets, as well as of tidal dissipation as a function of planetary radius.

This project spans the fields of high-performance computation, statistical modeling of experimental data, and celestial mechanics, which will make it a valuable contribution to the field of exoplanet studies.  We will release code and data using open-source collaboration tools, and help to guide the adoption of reproducible research standards by releasing interactive analysis packages as part of our publication process.

